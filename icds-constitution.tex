\documentclass[a4paper]{tufte-handout}

\usepackage{dashrule}
\usepackage{enumitem}
\usepackage{amsmath}
\usepackage{xfrac}
\usepackage{units}

\usepackage{draftwatermark}
\SetWatermarkText{Draft}
\SetWatermarkScale{1.5}

\title{Imperial College Dramatic Society Constitution}
% \author[The ICDS Committee]{The ICDS Committee}
\date{Proposed at the General Meeting on 03/11/15}

\newcommand{\policyOffset}{12pt}

\newcommand{\policyCspp}[2][\policyOffset]{\marginnote[#1]{\textsc{CSP Policy \S#2}}}
\newcommand{\policyCspEquip}[2][\policyOffset]{\marginnote[#1]{\textsc{CSP Equipment Policy \S#2}}}
\newcommand{\policyBye}[2][\policyOffset]{\marginnote[#1]{\textsc{Bye-laws \S#2}}}
\newcommand{\policyAE}[2][\policyOffset]{\marginnote[#1]{\textsc{A\&E Standing Orders \S#2}}}
\newcommand{\policyHs}[2][\policyOffset]{\marginnote[#1]{\textsc{Health \& Safety Policy \S#2}}}
\newcommand{\policyPub}[2][\policyOffset]{\marginnote[#1]{\textsc{Publicity Policy \S#2}}}
\newcommand{\policyTankard}[1][\policyOffset]{\marginnote[#1]{\textsc{Tankards Policy}}}

\addtolength{\hoffset}{24pt}

\begin{document}

\maketitle

\begin{fullwidth}
\itshape

\begin{abstract}
This constitution describes the structure of the Imperial College Dramatic Society and the procedures that govern it.
References to relevant sections of the Imperial College Union Constitution, Bye-laws and Policies are indicated alongside the text.
\end{abstract}
\end{fullwidth}

\hdashrule{15cm}{1.1pt}{1.1pt}

\section{Introduction}
\begin{enumerate}
    \item Imperial College Dramatic Society shall be a Society of the Imperial College Union and may hereafter be referred to as `the Society', `ICDS' or `DramSoc'.
    \item \policyCspp{33} The Society shall be a member of the Arts and Entertainments Management Group.
    \item `ICU' or the `Union' refers to Imperial College Union, and `IC' or the `College' refers to Imperial College London.
    \item The `Union Constitution' refers to the Constitution of the Union that is currently in force, as passed by the Union and College Councils and as may be amended from time to time according to \S12 of said Constitution.
    \item The `Union Bye-Laws' or `Bye-laws' shall refer to the \textit{Bye-laws of Imperial College Union} as passed by the Union and College Councils and as may be amended from time to time according to \S11 of the Union Constitution.
    \item The `CSP Policy' shall refer to the \textit{ICU Clubs, Societies and Projects Policy}.
    \item A Full Member of the Union is any Member of the Union who is not an Associate Member of the Union, as defined by \S13 and \S18 of the Union Constitution.
    \item Written notice, for the purpose of this document, shall be that as defined by \S114 of the Union Constitution.
\end{enumerate}

\section{Objectives}
\begin{enumerate}[resume]
    \item The Society aims to:
        \begin{enumerate}
            \item promote the furtherance of drama and the art of theatre in the Imperial College community.
            \item facilitate interaction with the student drama community outside of Imperial College.
            \item provide technical support for theatrical productions of DramSoc and of any other Club, Society or Project of the Arts and Entertainments Management Group.
            \item promote the furtherance of its Members' experience and excellence in the fields of theatre and other live events.
        \end{enumerate}
    \item The Society, as managed by its Committee, shall strive to achieve these objectives as its commitment to its membership.
\end{enumerate}

\section{Membership}
\begin{enumerate}[resume]
    \item Membership of the Society is split into \textsc{Full Members} of the Society and \textsc{Non-Voting Members} of the Society.
    \item \policyBye{B6} The voting membership of the Society is all of the Full Members of the Society.
    \item \policyCspp{67} Full Members of the Imperial College Union are exclusively able to become Full Members of the Society.
    \item \policyCspp{71} Associate Members of the Union may become Non-Voting Members of the Society.
    \item The Society shall have as its mascot a Cat, named ``Mistifer Cat''.
    \item ``Mistifer Cat'' shall be stuffed.
    \item Sir Walter Plinge and Mistifer Cat are offered honorary Non-Voting Membership of the Society.
    \item \policyCspp{17} Full and Associate Members of the Union may freely join the Society, subject to their paying any fee as may from time-to-time be stipulated by the Committee. If such a fee is charged, it shall be in accordance with any minimum specified in Union Policy.
    \item \policyAE{5.3,} \policyBye{B2} Union Members entitled to \textit{ex officio} membership are Non-Voting Members of the Society, unless they join the Society as a Full Member, satisfying all of the requirements placed thereupon.
\end{enumerate}

\section{Committee}
\begin{enumerate}[resume]
    \item Management of the Society, its funds and its property is vested in the Committee.
    \item The Committee shall have the power to resolve upon regulations and bye-laws as it sees fit, which may be enacted subservient to this constitution.
    \item All officers sitting on the Committee must be Members of the Society.
    \item \policyCspp{66} Any officer holding a voting position on the Committee must be a Full Member of the Society.
    \item \policyCspp{65} The Committee shall consist of at most thirteen voting officers holding the following offices:
        \begin{enumerate}
            \item President
            \item Vice-President
            \item Treasurer
            \item Honorary Secretary
            \item Acting Director
            \item Technical Director
            \item Lighting Director
            \item Sound Director
            \item Set, Props and Costumes Director
            \item Pub. Officer
            \item Social Secretary
            \item Systems Administrator
            \item Ordinary Committee Member
        \end{enumerate}
    \item Officers may hold more than one office, subject to the following restrictions:
        \begin{enumerate}
            \item Technical offices are defined to be Technical Director, Lighting Director, Sound Director, Set, Props and Costumes Director, and Systems Administrator.
            \item No officer holding one or more technical offices may hold a non-technical office.
            \item No officer may hold more than one of President, Vice-President, and Treasurer.
            \item The Ordinary Committee Member must not hold any other offices within the Society.
        \end{enumerate}
    \item \policyCspp{65} The term of office for all positions runs from the 1st of August to the 31st of July.
    \item A position of office may be vacated by an officer: \begin{enumerate}
            \item Resigning
            \item Reaching the end of their term of office
            \item Failing to hold full membership of the Society
            \item Becoming deceased
        \end{enumerate}
    \item Any officer ceasing to qualify for Full Membership of the Society must resign their position.
    \item \policyBye{G26,} \policyCspp{78} Should an office fall empty, a by-election shall be held at a General Meeting called by the President, which shall be subject to the same regulations as the elections held at the Annual General Meeting.
\end{enumerate}

\subsection{Meetings of the Committee}
\begin{enumerate}[resume]
    \item Meetings of the Committee shall be called by the President, and shall be held at least monthly during College term time.
    \item A meeting must be called by the President upon the request of any three members of the Committee.
    \item The President or their delegated authority shall chair meetings of the Committee.
    \item At least five College days' written notice of planned meetings of the Committee shall be provided by the Secretary to members of the Committee.
    \item Unplanned meetings may be held with shorter notice if there is a good reason to do so.
    \item Only officers of the Committee may attend Committee meetings, aside from exceptions made below.
    \item \policyBye{B11} Quorum of the Committee shall be 50\%+1 of the officers of the Committee.
    \item Any Member of the Society may observe Committee meetings as a Guest of the Committee.
    \item The Committee may invite other Members of the Union, or others outside of the Union as Guests of the Committee, as it sees fit.
    \item Guests of the Committee must ask permission from the Chair of the meeting to speak (ordinarily the President).
    \item The Secretary or their delegated authority shall circulate agenda for meetings of the Committee no less than one College day in advance of the meeting.
    \item The Secretary must circulate to the Committee a draft of the meeting's minutes within three college days of the meeting.
    \item A member may submit to the Secretary a request to view the draft non-reserved minutes. The Secretary must provide to such a member the draft minutes as soon as possible, so long as:
        \begin{enumerate}
            \item Three College days have passed since the circulation of the draft minutes to the Committee and guests,
            \item Before the minutes are provided to the member, any comments made on the draft by attendees to the meeting are appended
            \item The President has the right to veto the release of any section of the draft minutes until the following committee meeting, for reasons that must be provided to the member
            \item It is made clear that released minutes are unpublished and unapproved
        \end{enumerate}
    \item The Secretary must publish minutes of a meeting to the general public within three College days of the minutes' approval as a true and accurate record of the meeting to which they refer by the Committee.
    \item An archive of all non-reserved meeting minutes should be made publicly available by the Secretary.
    \item \textit{In extremis}, a member of the Committee may request that minutes be denoted as reserved, either in part or in full, and reasons for the reservation must be included in either the unreserved minutes, the reserved minutes, or both. An indication that minutes were reserved must be present in the unreserved minutes. A request to reserve minutes may be made at any time prior to the publication of the minutes publicly.
    \item Reserved minutes may be distributed only to members of the Committee, and, upon request, to the President of the Union, a Deputy President of the Union, or the Chair of the Society's Management Group.
    \item Whilst reserved minutes are being taken, Guests of the Committee may be asked to leave by the meeting's Chair.
    \item Committee members must strive to attend all Meetings of the Committee, and are obliged to submit apologies to the Secretary if they are unable to do so for whatever reason.
\end{enumerate}

\subsection{Subcommittees}
\begin{enumerate}[resume]
    \item The Committee may create further committees, without derogating from its responsibility, as its members see fit. Such committees shall be sub-committees of the Committee and shall be chaired and constituted as the Committee sees fit.
    \item Any member of the Committee shall be entitled to sit on any sub-committee so created, and may do so after informing the Committee as such.
    \item Such sub-committees shall not have decision making powers binding on the Society, but may make recommendations to the Committee.
    \item All members of sub-committees need only to be Members of the Union, not of the Society.
\end{enumerate}

\subsection{Officers}
\begin{enumerate}[resume]
    \item \newthought{The President:}
        \begin{enumerate}
            \item \policyCspp{8} shall be the senior officer of the Society.
            \item \policyCspp{83.i} shall be, along with the Treasurer, financially responsible for the Society.
            \item is responsible for ensuring that the Society is run according to this constitution.
            \item shall possess the casting vote in the event of a tie in a decision-making vote (as distinct from an election) at a Committee meeting or General Meeting of the Society.
            \item shall deputise or direct other members of the Committee to deputise for vacant positions, should this become necessary.
    \end{enumerate}
    \item \newthought{The Vice-President:}
        \begin{enumerate}
            \item shall deputise for the President in their absence.
        \end{enumerate}
    \item \newthought{The Treasurer:}
        \begin{enumerate}
            \item \policyCspp{83.i} shall be, along with the President, financially responsible for the Society.
            \item shall maintain accounts describing the Society's finances, in accordance with accepted accounting rules and practices.
            \item shall be responsible for ensuring the long-term financial viability of the Society, by holding reserves or instigating other structural measures as may be necessary to safeguard the objectives of the Society.
        \end{enumerate}
    \item \newthought{The Acting Director:}
        \begin{enumerate}
            \item shall have overall responsibility for the Society's acting, writing and directing activities.
            \item shall be responsible for ensuring that a Director can be found for the Society's plays, deputising if necessary.
        \end{enumerate}
    \item \newthought{The Technical Director:}
        \begin{enumerate}
            \item shall have overall responsibility for the Society's technical activities.
            \item shall ensure that the Society maintains an appropriate level of technical capability, in terms of both equipment and personnel, to enable the Society to achieve its objectives.
        \end{enumerate}
\end{enumerate}

\section{General Meetings}
\begin{enumerate}[resume]
    \item The President may call General Meetings of the Society.
    \item \policyCspp{74} A general meeting must be called by the President at the behest of any one of:
        \begin{enumerate}
            \item quorum of the Committee, or;
            \item a petition of ten or more Members made to the President in writing.
        \end{enumerate}
    \item General meetings may only be called during undergraduate term time, and must take place on, or in close vicinity of, the South Kensington campus.
    \item A General Meeting may not be called to start after 2100 or before 1000.
    \item No fewer than ten College days' notice of a general meeting must be given in writing to Members of the Society.
    \item Members may submit agenda for discussion at a general meeting to the Secretary no fewer than five College days before a general meeting. Any agendum so submitted must be proposed by a Member and seconded by at least two other Members.
    \item The Secretary must distribute the agenda for the General meeting to the Members of the Society in writing no fewer than four College days in advance of a general meeting.
    \item The first agendum of any general meeting must be to accept or reject by vote the minutes of the previous general meeting, or to note any corrections as upheld by vote if necessary.
    \item The Secretary or their delegated authority must record accurate minutes of the General meeting, which must be published in writing to the Members of the Society within 10 College days of the meeting. Attendees of the General meeting must be given the opportunity to comment on these minutes at least 3 college days before they are so published.
    \item The Secretary or their delegated authority must publish to the general public minutes which have been accepted at a General Meeting.
    \item Any Member of the Society may submit an opinion on any agenda to the Secretary \textit{in absentia}, which must be read out at the meeting by the Secretary in their stead.
    \item All decisions are made exclusively by the voting membership of the Society.
    \item \policyBye{B11} Quorum shall be 50\% + 1 of the full Members of the Society, who must be present in person or by appointed proxy.
\end{enumerate}

\subsection{Annual General Meeting}
\begin{enumerate}[resume]
    \item An Annual General Meeting (AGM) shall be called by the President during the Spring term.
    \item The principal business of the AGM shall be:
        \begin{enumerate}
            \item the presentation by each officer of their report of the preceding year.
            \item \policyCspp{75 \& \S77} the election of the officers of the Society, to be conducted as a `Minor Election' in accordance with {\S}G of the By-laws.
        \end{enumerate}
    \item \policyBye{G2} The Committee shall appoint a Returning Officer for the election.
    \item The Returning Officer:
        \begin{enumerate}
            \item may not stand for election.
            \item may not vote in the election.
            \item must be a Member of the Union.
            \item must open nominations for the election no fewer than five College days in advance of the AGM.
            \item must provide a publicly accessible form for the nomination and seconding of candidates.
            \item must distribute to all Members of the Society via the Secretary details of the arrangements for balloting, nomination and complaints no fewer than ten College days in advance of the AGM.
        \end{enumerate}
    \item Only voting Members of the Society may cast votes in the election.
    \item The vote to `Re-Open Nominations' shall be represented by Mistifer Cat in each election.
    \item Only Members who have been nominated by a Full Member of the Society and subsequently seconded by a different Full Member of the Society may stand for election.
    \item The nominator or the seconder of any candidate may not be the same person as the nominee.
    \item Candidates may be nominated and seconded at the AGM, so long as the above two conditions are satisfied.
    \item \policyBye{G14} Manifestos must be provided by all nominees, and may either be:
        \begin{enumerate}
            \item written, and submitted in advance in accordance with a timetable determined by the returning officer, or
            \item spoken, and presented by each nominee at the AGM.
        \end{enumerate}
\end{enumerate}

\section{Tankards}
\begin{enumerate}[resume]
    \item \policyTankard The Society must maintain Officer Tankards for the following positions:
        \begin{enumerate}
            \item President
            \item Treasurer
            \item Technical Director
        \end{enumerate}
    \item The Society may maintain other tankards, as so resolved upon by the Committee.
\end{enumerate}

\section{Constitution}
\begin{enumerate}[resume]
    \item \policyBye{B6,} \policyCspp{81} This document serves as the constitution for the Imperial College Dramatic Society.
    \item \policyCspp{36} Changes to this constitution may be made solely at a quorate General Meeting of the Society by a $\sfrac{2}{3}$ majority vote.
    \item This constitution shall be reviewed annually by the Committee.
    \item Any proposal to change this constitution must be made in writing to the Secretary as part of an agendum proposal for the General Meeting in question, who shall subsequently distribute the proposed change as an addendum to the agendum according to the schedule set out above.
    \item For the avoidance of doubt, since {\S}B16 of the bye-laws requires any electronic vote taken as a consequence of and subsequent to an inquorate meeting to obtain only a simple majority to pass, changes to this constitution cannot be ratified via this mechanism, and nor can any other decision that requires a two-thirds majority.
    \item This constitution is binding on the Society, the Committee, and its members.
    \item This constitution must be published publicly, and provided in full to any Member of the Union who requests it.
    \item All antecedent constitutions are hereby revoked.
\end{enumerate}

\section{Subjugation and Interpretation}
\begin{enumerate}[resume]
    \item This constitution shall be interpreted in accordance with the Law of England and Wales.
    \item The Committee shall be the arbiter in any question regarding the interpretation of this constitution.
    \item This constitution, and the Society, is bound by and defers to all superior bodies of the Union.
    \item The operation of the Society and resolutions of the Committee shall be in accordance with the Union Constitution, Bye-Laws, and all and any resolutions adopted as policy by any superior body, including:
        \begin{enumerate}
            \item The CSP Policy
            \item The Equal Opportunities Policy
            \item \policyCspp{33 \& \S36,} \policyAE{3.2} Any resolutions or standing orders of the Arts and Entertainments Management Group
        \end{enumerate}
    \item Nothing within this constitution shall be taken to conflict with any adopted Union Policy.
    \item If any superior body of the Union adopts or has adopted policy which causes any term of this constitution to be invalidated, the remainder shall continue to be held valid notwithstanding.
\end{enumerate}

\end{document}

% vim: expandtab:ts=4:sw=4
